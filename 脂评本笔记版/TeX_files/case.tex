\chapter{脂砚斋重评石头记凡例}

{\kaishu {《红楼梦》旨义} 是书题名极{[}多,一曰《红楼]\kaishu 梦》,是总其全部之名也;又曰《风月宝鉴》,是戒妄动风月之情;又曰《石头记》,是自譬石头所记之事也。此三名皆书中曾已点睛矣。如宝玉作梦,梦中有曲,名曰《红楼梦十二支》,此则《红楼梦》之点睛。又如贾瑞病,跛道人持一镜来,上面即錾``风月宝鉴''四字,此则《风月宝鉴》之点睛。又如道人亲眼见石上大书一篇故事,则系石头所记之往来,此则《石头记》之点睛处。然此书又名曰《金陵十二钗》,审其名,则必系金陵十二女子也;然通部细搜检去,上中下女子岂止十二人哉!若云其中自有十二个,则又未尝指明白系某某,及至``红楼梦''一回中,亦曾翻出金陵十二钗之簿籍,又有十二支曲可考。}

{\kaishu 书中凡写长安,在文人笔墨之间,则从古之称;凡愚夫妇、儿女子家常口角,则曰``中京'',是不欲着迹于方向也。盖天子之邦,亦当以中为尊,特避其东南西北四字样也。}

{\kaishu 此书只是着意于闺中,故叙闺中之事切,略涉于外事者则简,不得谓其不均也。}

{\kaishu 此书不敢干涉朝廷,凡有不得不用朝政者,只略用一笔带出,盖实不敢以写儿女之笔墨唐突朝廷之上也,又不得谓其不备。}

{\kaishu 此书开卷第一回也,作者自云:因曾历过一番梦幻之后,故将真事隐去,而撰此《石头记》一书也。故曰``甄士隐梦幻识通灵''。但书中所记何事?又因何而撰是书哉?自云:今风尘碌碌,一事无成,忽念及当日所有之女子,一一细推了去,觉其行止见识皆出于我之上,何堂堂之须眉诚不若彼一干裙钗?实愧则有馀、悔则无益之大无可奈何之日也。当此时,则自欲将已往所赖------上赖天恩,下承祖德,锦衣纨袴之时,饫甘餍美之日,背父母教育之恩,负师兄规训之德,以致今日一事无成、半生潦倒之罪,编述一记,以告普天下人。虽我之罪固不能免,然闺阁中本自历历有人,万不可因我不肖,则一并使其泯灭也。虽今日之茅椽蓬牖,瓦灶绳床,其风晨月夕,阶柳庭花,亦未有伤于我之襟怀笔墨者。何为不用假语村言敷演出一段故事来,以悦人之耳目哉?故曰``{[}贾雨村{]}风尘怀闺秀'',乃是第一回题纲正义也。开卷即云``风尘怀闺秀'',则知作者本意原为记述当日闺友闺情,并非怨世骂时之书矣。虽一时有涉于世态,然亦不得不叙者,但非其本旨耳。阅者切记之。}

{\kaishu 诗曰:}

{\kaishu  浮生着甚苦奔忙,盛席华筵终散场。}

{\kaishu  悲喜千般同幻渺,古今一梦尽荒唐。}

{\kaishu  谩言红袖啼痕重,更有情痴抱恨长。}

{\kaishu  字字看来皆是血,十年辛苦不寻常。}

% {\href{../Text/part0004.html\#navto_1_a}{①}此凡例五条及题诗仅见于甲戌本卷首,退二格抄写。其他各本均无凡例,且均截取第五条``此开卷第一回也''并入第一回作为正文开始。}
% {\href{../Text/part0004.html\#navto_2_a}{②}此处原被撕去一角,缺五字。胡适补书``多''``红楼''三字,吴恩裕另校补``一曰''两字。除``红楼''二字无争议外,前三字所补是否恰当,有不同意见。}
