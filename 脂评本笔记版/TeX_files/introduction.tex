\chapter{整理说明}

《红楼梦脂评汇校本》,系以甲戌本、己卯本、庚辰本等早期脂本为底本,汇集了戚序本、蒙府本等其他脂批本的部分脂批,并参考、吸收若干新校点本及脂批辑本的校点成果整理而成。对前人意见有分歧的,略参己意而取舍,力求既不人云亦云,也不标新立异,整理成为一个方便阅读和检索的脂评红楼梦简明读本。

本书正文以甲戌本及庚辰本为底本(第六十四、六十七回以列藏本为底本),以其他各脂本参校。甲戌本所存十六回,文字显著优于他本,基本照录原文,只在确有必要时改动少量字词;庚辰本内容比较完整,惜抄写草率,错误甚多,不得不参照己卯本等本子校改大量字句。第六十七回两种版本文字差异过大,无法互校,一并附录。

本书辑录现存各脂本上的固有批语,并剔除在抄本流传过程中收藏者或读者所加的批语。批语按以下顺序辑录:甲戌本、己卯本、庚辰本、戚序本、蒙府本、列藏本、甲辰本。为节省篇幅,后出版本的批语与前面某本文字相同的,不再列出;有个别文字差异的,只据以参校,也不单独列出。

为节省篇幅,本书辑录批语使用略字,意义如下:\includegraphics[width=3mm]{../Images/00002}(甲戌本)、\includegraphics[width=3mm]{../Images/00003}(己卯本)、\includegraphics[width=3mm]{../Images/00004}(庚辰本)、\includegraphics[width=3mm]{../Images/00005}(戚序本)、\includegraphics[width=3mm]{../Images/00006}(蒙府本)、\includegraphics[width=3mm]{../Images/00007}(列藏本)、\includegraphics[width=3mm]{../Images/00008}(杨藏本)、\includegraphics[width=3mm]{../Images/00009}(甲辰本);\includegraphics[width=3mm]{../Images/00010} (眉批。原抄在页眉{[}正文上边{]}的批语)、\includegraphics[width=3mm]{../Images/00011} (侧批。原抄在正文右侧的批语)、\includegraphics[width=3mm]{../Images/00012}  (夹批。原抄在正文中间的双行批语)。为醒目起见,本书正文排为宋体字,批语排为楷体小号字:朱批仍用红色,墨批仍用黑色。抄本中有些同一位置不止一条批语的,原以``○''或空格隔开,现统一以``◇''隔开。

本书使用半角圆括号()和半角方括号{[}{]}作为校改文字记号:(某)表示删字,{[}某{]}表示补字,(甲){[}乙{]}表示改字。为避免校改记号过多影响阅读,对明显错字及前人意见比较一致的校改成果,径改而不标记。

本书曾以电子版的形式在网络上广泛流传,产生了一定的影响。在电子版整理过程中,抚琴居网站的朋友们浪知、君子九思、爱如潮水、娄员外、梁三、兔子她妈、休闲、志学斋主、daphne等协助覆校,订正不少错误;冰冰冰冷、云涛、yupeng、蓝山、潇湘楚客、zhwl、liuzhoufish、寻梦园、阿迦、影乐之声、rocwings等诸位先后指出了若干问题,统此鸣谢。在电子版传播过程中,许多朋友提出了出版纸版书的建议,还有一些朋友自行打印装订成册,以方便阅读使用。今天本书终得印行,要特别感谢上海《红楼梦研究辑刊》主编萧凤芝老师、白山出版社原总编董志新老师的大力支持和协助;感谢八一电影制片厂高级美工师汪德龙老师为本书题签。

由于整理者学识所限,本书虽经多次修订,存在问题必然尚多。希望朋友们在使用过程中发现错误及不妥之处,继续不吝赐教。

\begin{flushright}
	吴铭恩 ~2013.9.30.
\end{flushright}~~~~
